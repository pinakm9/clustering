\documentclass{article}
\usepackage[utf8]{inputenc}
\usepackage{hyperref}
\hypersetup{
    colorlinks=true,
    linkcolor=blue,
    filecolor=magenta,
    urlcolor=blue,
}
\title{Clustering Demos}
\author{Pinak, Srashti }
\date{March 2020}

\begin{document}

\maketitle

\section*{Document classification with Hierarchical clustering}
\subsection*{Dataset}
20 newsgroups dataset available in  \href{https://scikit-learn.org/0.19/datasets/twenty_newsgroups.html}{scikit}
\subsection*{Notebook}
1. Load and look at sample data
\newline
2. Filter text (section 5.6.2.3) and look at the filtered text
\newline
3. Vectorize text
\newline
4. HAC with \href{https://scikit-learn.org/stable/modules/generated/sklearn.cluster.AgglomerativeClustering.html}{sklearn.cluster} or \href{https://joernhees.de/blog/2015/08/26/scipy-hierarchical-clustering-and-dendrogram-tutorial/}{scipy.cluster}
\newline
5. Plot truncated dendrogram
\newline
6. Figure out the number of clusters based on \href{https://en.wikipedia.org/wiki/Silhouette_(clustering)}{silhouettes}
\newline
7. Figure out the number of misclassifications
\subsection*{To Learn}
1. Hierarchical Clustering
\newline
2. Bag of words model and \href{https://towardsdatascience.com/natural-language-processing-feature-engineering-using-tf-idf-e8b9d00e7e76}{TF-IDF}
\newline
3. Silhouettes
\newline
4. Manipulation of 20 newsgroups dataset
\newline
5. Using Python to implement HAC
\section*{Edge detection with k-means}
\subsection*{Dataset} Images collected off the web

\subsection*{Notebook}
1. Load and create greyscale image
\newline
2. Show that greyscaling preserves the edges
\newline
3. Compute features
\newline
4. Use k-means to find edge pixels
\newline
5. \href{https://stackoverflow.com/questions/434583/what-is-the-fastest-way-to-draw-an-image-from-discrete-pixel-values-in-python}{Create an image from edge pixels using Pillow}
\subsection*{To Learn}
1. k-means
\newline
2. \href{https://stackoverflow.com/questions/12201577/how-can-i-convert-an-rgb-image-into-grayscale-in-python}{Greyscaling an image in Python using Pillow}
\newline
3. What are the features?
\newline
4. How to detect an edge pixel?
\newline
5. Creating an image from edge pixels
\section*{Image Segmentation with DBSCAN}
\subsection*{Dataset}
Images collected off the web
\subsection*{Notebook}
\end{document}
