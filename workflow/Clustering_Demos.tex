\documentclass{article}
\usepackage[utf8]{inputenc}
\usepackage{hyperref}
\hypersetup{
    colorlinks=true,
    linkcolor=blue,
    filecolor=magenta,
    urlcolor=blue,
}
\title{Clustering Demos}
\author{Pinak, Srashti }
\date{March 2020}

\begin{document}

\maketitle

\section*{Document classification with Hierarchical clustering}
\subsection*{Dataset}
20 newsgroups dataset available in  \href{https://scikit-learn.org/0.19/datasets/twenty_newsgroups.html}{scikit}
\subsection*{Notebook}
1. Load and look at sample data
\newline
2. Filter text (section 5.6.2.3) and look at the filtered text
\newline
3. Vectorize text
\newline
4. HAC with \href{https://scikit-learn.org/stable/modules/generated/sklearn.cluster.AgglomerativeClustering.html}{sklearn.cluster} or \href{https://joernhees.de/blog/2015/08/26/scipy-hierarchical-clustering-and-dendrogram-tutorial/}{scipy.cluster}
\newline
5. Plot truncated dendrogram
\newline
6. Figure out the number of clusters based on \href{https://en.wikipedia.org/wiki/Silhouette_(clustering)}{silhouettes}
\newline
7. Figure out the number of misclassifications
\subsection*{To Learn}
1. Hierarchical Clustering
\newline
2. Bag of words model and TF-IDF
\newline
3. Silhouettes
\newline
4. Manipulation of 20 newsgroups dataset
\newline
5. Using Python to implement HAC
\section*{Edge detection with k-means}
\subsection*{Dataset} Images collected off the web

\subsection*{Notebook}
1. Load and create greyscale image
\newline
2. Compute features
\newline
3. Use k-means to find edges

\end{document}
